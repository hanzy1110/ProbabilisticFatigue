% Created 2024-01-09 mar 15:30
% Intended LaTeX compiler: pdflatex
\documentclass[11pt]{article}
\usepackage[utf8]{inputenc}
\usepackage[T1]{fontenc}
\usepackage{graphicx}
\usepackage{longtable}
\usepackage{wrapfig}
\usepackage{rotating}
\usepackage[normalem]{ulem}
\usepackage{amsmath}
\usepackage{amssymb}
\usepackage{capt-of}
\usepackage{hyperref}
\author{bm69}
\date{\today}
\title{Calculo de la Probabilidad de falla en Conductores ACSR}
\hypersetup{
 pdfauthor={bm69},
 pdftitle={Calculo de la Probabilidad de falla en Conductores ACSR},
 pdfkeywords={},
 pdfsubject={},
 pdfcreator={Emacs 27.1 (Org mode 9.7)}, 
 pdflang={English}}
\begin{document}

\maketitle
\tableofcontents

\section{Motivación}
\label{sec:org7ad283e}
Por que es importante conocer la probabilidad de falla en los Conductores ACSR
\section{Detalle del proceso de fatiga en conductores eléctricos.}
\label{sec:org696e3d5}
\begin{itemize}
\item Que constituye la falla por fatiga.
\begin{itemize}
\item Modelo de Poffenberger \& Swart
\item Mas antecedentes del trabajo de fatiga en conductores
\end{itemize}
\item Como es el proceso de falla en los conductores
\item Consecuencias de las fallas por fatiga
\end{itemize}
\section{Conceptos del proceso de fatiga}
\label{sec:orgf971f1f}
\subsection{Fatiga deterministica. Conceptos clave}
\label{sec:org27b104f}
\subsection{Fatiga estocástica. Conceptos, requisitos}
\label{sec:orgb54aa2d}
\begin{itemize}
\item Cargas aleatorias => Conteo de ciclos
\item Propiedades aleatorias => Curvas de Wohler
\end{itemize}
\subsection{Calculo de la probabilidad de falla}
\label{sec:org4f1d813}
\begin{itemize}
\item Mechar con la teoría del polaco
\end{itemize}
\section{Confiabilidad}
\label{sec:org8749d57}
\begin{itemize}
\item Que es la Confiabilidad
\item Confiabilidad en relación a los modelos físicos
\begin{itemize}
\item Mechar con el proyecto de investigacion
\end{itemize}
\item La relación de la confiabilidad con el fenómeno de fatiga
\item El daño acumulado como medida de la confiabilidad
\end{itemize}
\subsection{Estimación del daño}
\label{sec:org14a3beb}
\begin{itemize}
\item Modelado del proceso de fatiga
\begin{itemize}
\item Tipo de carga
\end{itemize}
\item Distintas técnicas para el modelado del daño acumulado
\begin{itemize}
\item Regla de Miner
\item Modelo de Aeran
\end{itemize}
\item Tengo una familia de modelos
\end{itemize}
\section{Fuentes de incerteza y formas de estimarla:}
\label{sec:org1197d19}
\subsection{Curva de Wohler:}
\label{sec:orgbff497f}
\begin{itemize}
\item Fuentes de incerteza experimental
\item Problemas al ensayar conductores.
\item Problemas de la variancia variable.
\item Calculo de modelo probabilístico Heterocedastico
\end{itemize}
\subsection{Proceso de carga:}
\label{sec:org785c917}
\begin{itemize}
\item Mediciones en campo
\item Conteo de ciclos (Funcionamiento del Vibrec)
\item Modelo probabilístico para la distribución de las amplitudes de oscilación
\end{itemize}
\subsection{Calculo de los esfuerzos:}
\label{sec:orgb6ff10a}
\begin{itemize}
\item Incerteza en el modelado del esfuerzo de fatiga
\item Formas de solucionar el problema
\item Tratamiento estadístico de la estimación de los esfuerzos por flexión.
\item Uso de la inferencia Bayesiana para estimar la incerteza del modelo
\end{itemize}
\section{Modelado del proceso de fatiga para el calculo de la probabilidad de falla}
\label{sec:orgbf8c7d1}
\subsection{Objetivos:}
\label{sec:org44fc1a9}
\begin{itemize}
\item Estudiar la evolución de la probabilidad de falla a partir de mediciones iniciales
\item Estudiar un proceso de fatiga anual a partir de las mediciones del campo
\item Extrapolar el proceso a lo largo del periodo estimado.
\end{itemize}
\subsection{Implementación}
\label{sec:org0ea036e}
\begin{itemize}
\item Sampleo de los posibles fenómenos de fatiga a partir de las distribuciones calculadas para amplitud y esfuerzo
\begin{itemize}
\item Obtener cargas y cantidades de ciclos a una determinada carga
\item La distribución del daño viene de calcular un gran numero de procesos de fatiga distintos y a partir de eso estimar el daño
\end{itemize}
\item Sampleo de la vida a la fatiga en función de las cargas obtenidas
\item Combinación de los tres elementos en el modelo de danio
\item Obtención de la probabilidad de falla a partir de las distribuciones anuales y la transferencia del daño entre años
\end{itemize}
\end{document}
